\documentclass[11pt,a4paper]{amsart}


%% package
\usepackage[T1]{fontenc}
\usepackage[utf8]{inputenc}
\usepackage[french]{babel}
%\usepackage[dvips]{graphicx}
\usepackage{amsmath, amsthm}
\usepackage{amsfonts,amssymb, amsbsy}
\usepackage{multicol} 
\usepackage{stmaryrd}
\usepackage{longtable}
\usepackage{enumerate}
\usepackage{mathrsfs}
\usepackage[pdftex]{graphicx}


%\usetikzlibrary{arrows}


\usepackage{pgf,tikz}
\usetikzlibrary{arrows}

\usepackage{color}
\long\def\red#1{\textcolor {red}{#1}}
\long\def\blue#1{\textcolor {blue}{#1}}


\theoremstyle{theorem}
\newtheorem{thm}{Th\'eor\`eme}
\newtheorem{lemmme}[thm]{Lemme}

\newtheorem{theoreme}[thm]{Th\'eor\`eme}
\newtheorem*{theorem*}{Th\'eor\`eme}
\newtheorem{proposition}[thm]{Proposition}


\theoremstyle{definition}
\newtheorem{definition}[thm]{D\'efinition}
\newtheorem{remarque}[thm]{Remarque}
\newtheorem{exemple}[thm]{Exemple}
\newtheorem{corollaire}[thm]{Corollaire}
\newtheorem*{attention}{Mise en garde}

\newtheorem{notation}[thm]{Notation}


\newcommand{\coeur}{$\heartsuit$}
\newcommand{\ccoeur}{$\heartsuit\heartsuit$}
\newcommand{\trefle}{$\clubsuit$}
\newcommand{\ttrefle}{$\clubsuit\clubsuit$}

\newenvironment{preuve}{\smallskip\noindent\emph{\textbf{Preuve.}}\hspace{1pt}}%
 {\hspace{-5pt}{\nobreak\quad\nobreak\hfill\nobreak$\square$\vspace{8pt}%
 \par}\smallskip\goodbreak}
%% macro


\DeclareMathOperator{\dimension}{dim}
\DeclareMathOperator{\kernel}{ker}
\DeclareMathOperator{\image}{image}










\def\norm#1{\Vert #1 \Vert }
\def\scal#1{\langle #1\rangle}





\def\siecle#1{\textsc{\romannumeral #1}\textsuperscript{e}~si�cle}
\newcommand{\eps}{\varepsilon}

\newcommand{\module}[1]{\vert #1 \vert}
%\frenchbsetup{StandardItemLabels}

\renewcommand{\Re}{\mathrm{Re}}
\renewcommand{\Im}{\mathrm{Im}}



\newcommand{\dd}{d}

\def\NN{{\mathbb N}}
\def\ZZ{{\mathbb Z}}
\def\RR{{\mathbb R}}
\def\CC{{\mathbb C}}
\def\QQ{{\mathbb Q}}

\def\PP{{\mathbb P}}
\def\EE{{\mathbb E}}
%\def\S{{\mathbb S}}
%\def\O{{\mathbb O}}
%\def\F{{\mathbb F}}
%\def\K{{\mathbb K}}
%\def\M{{\mathcal M}}
%\def\S{{\mathfrak{S}}}
%\def\A{{\mathfrak{A}}}
%\def\U{{\mathbb{U}}}
%\def\ZN{{$\Z/n\Z$}\xspace}
%\def\eps{\varepsilon}
%\def\codim{\mathrm{codim}}
\def\card{\mathrm{Card}}
%\def\rg{\mathrm{rg}}
%\newcommand{\cqfd}{\hfill $\square$ \medskip}
%\newcommand{\re}{\Re\mathrm{e}}
%\newcommand{\im}{\mathrm{Im}}
%\newcommand{\ch}{\mathrm{ch}}
%\newcommand{\sh}{\mathrm{sh}}
%\newcommand{\tth}{\mathrm{th}}
%\newcommand{\tq}{~/~}



%%compteurs
%\setlength{\parindent}{0pt}
\newcounter{numExercice}
\newcounter{NumPartie}
\newcounter{qcounter}
\newcounter{qscounter}
\setcounter{numExercice}{0}
\setcounter{NumPartie}{0}
\setcounter{qcounter}{0}
\setcounter{qscounter}{0}



%%Exercices, problemes, questions
\newcommand{\questiondecours}[1]{\setcounter{qcounter}{0}\vspace{.5cm}\hangindent0em\hangafter=0%
\noindent\textbf{Question de cours.}~}

\newcommand{\exercicesansnumero}[1]{\setcounter{qcounter}{0}\vspace{.5cm}\hangindent0em\hangafter=0%
\noindent\textbf{Exercice.}~}

\newcommand{\exo}[2]{\vspace{.3cm}\hangindent0em\hangafter=0%
\noindent{\textbf{Exercice(#2).} \textit{#1}}}


\newcommand{\exosd}[1]{\vspace{.3cm}\hangindent0em\hangafter=0%
\noindent{\textbf{Exercice .} \textit{#1}}}

\newcommand{\pb}[1]{\vspace{.6cm}
\noindent\textbf{Probl�me  .}
{\textit{{#1}}}
}

\newcommand{\partie}[1]{
\stepcounter{NumPartie}\setcounter{qcounter}{0}
\vspace{.3cm}
\begin{center}
\textbf{Partie \Roman{NumPartie}. #1}
\end{center}
}


%\newcommand{\solution}[1]{\ifx\withsolution\undefined\else{\bigskip \noindent\it {\bf Solution~: }#1\cqfd}\fi}

%\newcommand{\exercice}[1]{\subsection{#1}\setcounter{qcounter}{0}}

\newenvironment{question}
{
  \stepcounter{qcounter}
  \setcounter{qscounter}{0}
  \vspace{.3cm}
  \hangindent1em
  \hangafter=0
  {\noindent(\arabic{qcounter}) }
}%
{
}

\newenvironment{squestion}
{%
  \stepcounter{qscounter}%
  \hangindent3em%
  \hangafter=0%
  \vspace{.1cm}
  {\noindent(\alph{qscounter})}
}%



\renewcommand{\d}[1]{\mathinner{\mathrm{d}{#1}}}

\newcommand{\modulo}[1]{\vert #1\vert}
\DeclareMathOperator{\id}{id}

\begin{document}
\everymath{\displaystyle}$\textbf{Exercice 6}$


\question On a vu dans un exercice que la fonction $g:x\mapsto \sin(2x)-2x$ était strictement décroissante sur $\RR^+$. En déduire que l'on a $\sin(2x)<2x$ pour tout $x>0$.


\question Faire l'étude complète de la fonction $f:x\mapsto \cos(2x)+2x^2$

\textit{(Penser à utiliser la question 1) pour étudier le sens de variation !)}

\question Démontrer que pour tout $x\in \RR$ on a $\cos(2x)\geq 1-2x^2$.


% ------------------------------------------------------------------------------------------------------------------------------------------------------------------------------------
$\textbf{Exercice 53}$

Soit $f$ la fontion définie par $$f(x)=\frac12\ln(\frac{1+x}{1-x})$$

\question Donner le domaine de définition $D_f$ de $f$. La fonction $f$ est-elle paire, impaire, ou aucun des deux ?

\question

\squestion Expliquer pourquoi on a $f(x)=\frac{1}{2}(\ln(1+x)-\ln(1-x))$ (et en particulier vérifier que cette expression est bien définie pour $x\in D_f$..

\squestion En déduire que l'on a $$f'(x)=\frac{1}{1-x^2}$$

\question Donnez les limites de $f$ au bord du domaine de définition.

\question Faites le tableau de variation de $f$ et tracez son graphe.



\question En utilisant les questions précédentes, résoudre sur $]-1,1[$ l'équation $(1-x^2)y'-y=0$

\question Bonus : quelles sont les solutions de l'équation $(1-x^2)y'-y=0$ sur $]1,+\infty[$ et sur $]-\infty,-1[$ ?

\question On pose $y_0(x)=\frac{1+x}{\sqrt{1-x}}$

 \squestion Calculer la dérivée de $y_0$.
 
\squestion Démontrer que $y_0$ est une solution de l'équation 
$$y'-\frac{1}{1-x^2} y=\frac{1}{2\sqrt{1-x}}$$

\squestion En déduire toutes les solutions sur $]-1,1[$ de l'équation $$y'-\frac{1}{1-x^2} y=\frac{1}{2\sqrt{1-x}}$$

$\textbf{Exercice 45}$\exosd{5min}

Soit $f$ une fonction. 

\question Quelle propriété de $f$ peut s'écrire avec la phrase suivante :

$$\forall A>0\;\exists \eta>0\forall x\; |x-2|<\eta\Rightarrow f(x)>A$$

\question Faire la négation de la phrase ci-dessus.

$\textbf{Exercice 78}${\ccoeur}
Placer dans le plan les points dont les affixes sont :
\begin{enumerate}
\begin{multicols}{2}
\item $2i$
\item $-1$
\item $1+i$
\item $2+3i$
\item $3-5i$
\item $\sqrt{3}/2+i/2$
\item $\cos(\pi/2)+i\sin(\pi/2)$
\item $i(i+1)$
\end{multicols}
\end{enumerate}

$\textbf{Exercice 96}${\coeur} Déterminer la limite de $f$ en $-\infty$ et en $+\infty$ dans chacun des cas suivants : 

\begin{multicols}{2}
\begin{enumerate}
\item $f(x)=e^{2x}-e^x$,
\item $f(x)=\frac{e^{2x}-1}{e^x+1}$,
%\item $f(x)=x(\sqrt{e^{2x}+1}-e^x)$,
\item $f(x)=e^{x^2}-e^{x+1}$.
\end{enumerate}
\end{multicols}

$\textbf{Exercice 152}${\trefle}
Calculez dans chaque cas une primitive des fonctions définies par les formules suivantes. Vous  aurez parfois besoin d'une intégration par parties, parfois pas.

\begin{multicols}{2}
\begin{enumerate}[a)]
\item $x\cos(x)$ (dériver $x$ et intégrer $\cos$)
\item $x^2\sin(x)$ (se ramener à l'exemple précédent)
\item $x^2e^x$ (se ramener à $xe^x$)
\item $\sin(5x)-2\cos(3x)+4\sin(2x)$ 
\item $\frac{x}{e^x}$
\item $\frac{x}{(x^2-4)^2}$
\item $(2x+1)e^x$,
\item $\sin(x)\cos(x)^3$
\item $x\ln(x)$
\item $x^2\ln(x)$
\item $\cos(x)^2$
\end{enumerate}
\end{multicols}

\section{Equations différentielles}

$\textbf{Exercice 5}$

\question 
Faire l'étude complète de la fonction $x\mapsto \cos(2x)-x$.


\question 
Faire l'étude complète de la fonction $x\mapsto \cos(2x)+2x^2$. 

\textit{On pourra utiliser le fait (démontré dans un exercice posé en classe) que pour tout $x>0$ on a $\sin(2x)<2x$}

\question En déduire que pour tout $x\in\RR$ on a $\cos(2x)<1-2x^2$

% ------------------------------------------------------------------------------------------------------------------------------------------------------------------------------------
$\textbf{Exercice 156}$\begin{exo}{}{}
Calculez les dérivées des fonctions $f$ définies par:


\begin{enumerate}[(a)]
\item $  f(x)=\frac{e^x \ln(x) }{x^2+2x^3}$,%$f'(x) = \frac{e^x ((2 x^2-5 x-2) \ln(x)+2 x+1)}{x^3 (2 x+1)^2}$
\item $ f(x)= 3^x \sin x $,% $f'(x)=3^x (\cos(x)+\ln(3) \sin(x))$
\item $ f(x)= \frac{x\ln x}{x^2-1}$, %$f'(x)= \frac{x^2-(x^2+1) \ln(x)-1}{x^2-1}^2$.
\item $f(x)=\frac{4\cos^2(x)-3}{2\cos(x)}$, %$f'(x) = -2 \sin(x)- \frac{3}{2} \frac{\sin(x)}{\cos(x)^2}$
\item $f(x)=(x^4-x^2+5)^4$, %$f'(x) = 8 x (2 x^2-1) (x^4-x^2+5)^3$
\item $f(x)=\frac{1}{\sqrt{x-3}}$, %$f'(x) = -\frac{1}{2 (x-3)^{3/2}}$
\item $f(x)=(\ln(x))^3$,% $f'(x) = 3 \frac{\ln^2(x)}{x} $
\item $f(x)=(x^2+x+1)e^x$, %$f'(x) = e^x (x^2+3 x+2) $
\item $f(x)=\frac{-3x^2+4x-1}{x^2+2x+5}$, %$f'(x) = -2 \frac{5 x^2+14 x-11)}{x^2+2 x+5}^2$
\item $f(x)=(1-x)\sqrt{x+1}$, %$f'(x) = \frac{-3 x-1}{2 \sqrt{x+1}} $
\item $f(x)=\ln(\frac{2x-1}{x-3})$,% $f'(x) = -\frac{5}{2 x^2-7 x+3}$
\item $f(x)=\sqrt{(\ln(x))^2 +1}$,% $f'(x) = \frac{\ln(x)}{x \sqrt{\ln^2(x)+1}}$
\item $f(x)=\ln(e^{2x}-1)$, %$f'(x) = \frac{(2 e^{2 x}}{e^{2 x}-1} $
\item $f(x)=\ln(x+\sqrt{x^2+1})$, %$f'(x) = \frac{1}{\sqrt{x^2+1}}$
\item $f(x)=x\sqrt{\frac{x-1}{x+1}}$
\end{enumerate}
\end{exo}


\section{Complexes}

$\textbf{Exercice 211}${\ccoeur}

Donner un équivalent simple au voisinage du point demandé :

\begin{enumerate}[a)]
\begin{multicols}{2}
\item en $0$, $\exp(\sqrt{1+x})$
\item en $+\infty$, $\frac{5x^3+3\ln(x)+2x}{4x^4+3x^2+1}$
\item en $0$,  $\frac{5x^3+3\ln(x)+2x}{4x^4+3x^2+1}$
\item en $1$, $\ln(x)$
\item en $+\infty$, $\ln(x+1)-\ln(x)$
\end{multicols}
\end{enumerate}


$\textbf{Exercice 117}${\coeur}

\question Ecrivez (sans tricher !) les symboles de l'alphabet grec sur votre feuille : dans l'ordre,

{alpha},{bêta},{gamma},{delta},{epsilon},{zêta},{êta},{thêta},
{iota},kappa,

lambda,mu,nu,xi,omicron,pi,rhô,sigma,tau,upsilon,phi,khi,
psi,
oméga

\question Cachez l'énoncé, et lisez ce que vous avez écrit sur votre feuille.


$\textbf{Exercice 70}${\coeur}

Soit $f$ une fonction définie sur $\RR$. Que signifient les assertions suivantes ? Dans chaque cas, donner un exemple de fonction vérifiant la propriété, et un exemple ne la vérifiant pas.

\begin{enumerate}
\item $\forall A>0\; \exists B>0 \;\forall x>B\quad f(x)>A$
\item $\forall \eps>0\; \exists A>0\;\forall x>A\quad |f(x)|<\eps$.
\item $\forall \eps>0\;\exists A>0\;\forall x>A\quad |f(x)-1|<\eps$.
\item $\exists l \in \RR \; \forall \eps>0\;\exists A>0\;\forall x>A\quad |f(x)-l|<\eps$.
\item $\forall \eps>0 \; \exists \eta >0 \; \forall x\in\RR \; \quad |x|<\eta \Rightarrow |f(x)-1|<\eps$.
\item $\forall \eps >0 \; \exists \eta \in \RR, \forall x \in \RR, |x-1|<\eta \Rightarrow |f(x)- 2|< \eps$.
\item Soit $ x_0 \in \RR. \forall \eps >0 \; \exists \eta \in \RR, \forall x \in \RR, |x-x_0|<\eta \Rightarrow |f(x)-f(x_0)|< \eps$.
\item $\forall x_0 \in \RR \; \forall \eps >0 \; \exists \eta \in \RR, \forall x \in \RR, |x-x_0|<\eta \Rightarrow |f(x)-f(x_0)|< \eps$.
\end{enumerate}


$\textbf{Exercice 67}${\coeur}

On considère la phrase "Pour tout nombre réel $x$, il existe un entier naturel $N$ tel que $N>x$.

\question Traduire cette phrase à l'aide de quantificateurs.

\question \'Ecrire sa négation en français et avec des quantificateurs.


$\textbf{Exercice 65}${\trefle}
Soit $f:\RR\to\RR$ une fonction. Voici plusieurs propriétés possibles de la fonction $f$. Quelles sont, en langage courant, leur signification ? Dans chaque cas, pouvez-vous trouver une fonction $f$ qui vérifie cette propriété ? Et une autre qui ne la vérifie pas ?

\begin{enumerate}[(i)]
\item $\forall x\in \RR\; \exists y\in \RR \;\; f(x)<f(y)$
\item $\forall x\in \RR \; \exists T\in \RR \;\; f(x)=f(x+T)$
\item $\forall x\in \RR \;\exists T\in \RR^* \;\; f(x)=f(x+T)$
\item $\forall x\in \RR \; \exists y\in \RR \;\; f(x)=y$
\item $\exists x\in \RR \; \forall y\in \RR\;\; f(x)=y$
\end{enumerate}

$\textbf{Exercice 64}${\coeur}

Soit $f:\RR\to\RR$ une fonction. 

\question Exprimer à l'aide de quantificateurs les assertions suivantes :

\begin{enumerate}[(i)]
\item $f$ est croissante
\item $f$ est impaire
\item $f$ est constante
\item $f$ est périodique de période $2\pi$
\item $f$ n'est ni croissante ni décroissante
\item $f$ est injective
\item $f$ est surjective
\end{enumerate}

\question \'Ecrire leur négation.




$\textbf{Exercice 57}${\coeur}

Dire si les propositions suivantes sont
vraies ou fausses, et les nier.

\begin{enumerate}
\item Pour tout réel $x$, si $x\geq 3$ alors $x^2\geq 5$.
\item Pour tout entier naturel $n$, si $n>1$ alors $n\geq 2$.
\item Pour tout réel $x$, si $x>1$ alors $x\geq 2$.
\item Pour tout réel $x$,  $x^2\geq 1$ est équivalent à $x\leq 1$.
\end{enumerate}

$\textbf{Exercice 55}${\coeur}

Soit $x\in\RR$. Nier les propositions suivantes :

\begin{enumerate}[a)]
\item $0\leq x\leq 1$
\item $x=0$ ou $(x\geq 0$ et $x^2=1$)
\item $\forall y\in\RR$, $xy\neq 0$ ou $x=0$ ou $y=0$.
\end{enumerate}

Dans chaque cas, sont-elles vraies ou fausses ?

\end{document}